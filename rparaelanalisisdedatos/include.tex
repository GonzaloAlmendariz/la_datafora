\usepackage{pdfpages}  
% --- Encabezado profesional para Quarto libro PDF ---

% Tipografía profesional
\usepackage{libertinus}     % Fuente principal: Libertinus Serif + Libertinus Math
\usepackage{microtype}      % Mejora justificación y espaciado fino

% Encabezados y pies de página
\usepackage{fancyhdr}
\pagestyle{fancy}
\fancyhf{} % Limpia encabezado y pie

% Encabezados
\fancyhead[LE]{\textit{R PARA EL ANÁLISIS DE DATOS}} % Página par (izquierda): nombre del libro
\fancyhead[RO]{\leftmark}                            % Página impar (derecha): capítulo

% Pies de página: número alineado al margen
\fancyfoot[LE]{\thepage} % Página par: número a la izquierda
\fancyfoot[RO]{\thepage} % Página impar: número a la derecha

% --- Tipografía y espaciado general ---

% Interlineado ligeramente ampliado para lectura cómoda
\usepackage{setspace}
\setstretch{1.1}

% Prevención de viudas y huérfanas
\clubpenalty=10000
\widowpenalty=10000
\displaywidowpenalty=10000

% Títulos de capítulos con estilo claro
\usepackage{titlesec}
\titleformat{\chapter}[display]
  {\normalfont\Huge\bfseries}
  {\chaptertitlename\ \thechapter}{20pt}{\Huge}

% Espaciado de secciones
\titleformat{\section}
  {\normalfont\Large\bfseries}{\thesection}{1em}{}
\titleformat{\subsection}
  {\normalfont\large\bfseries}{\thesubsection}{1em}{}

% Notas al pie más limpias
\usepackage[hang,flushmargin]{footmisc}
\renewcommand{\footnotelayout}{\normalsize}

% --- Ajustes para listas ---
\usepackage{enumitem}
\setlist[itemize]{topsep=3pt, partopsep=0pt, itemsep=3pt, parsep=0pt}
\setlist[enumerate]{topsep=3pt, partopsep=0pt, itemsep=3pt, parsep=0pt}




% Soporte manual para caracteres Unicode tipo bloque (sparklines)
\DeclareUnicodeCharacter{2581}{\rule{1ex}{0.2ex}} % ▁
\DeclareUnicodeCharacter{2582}{\rule{1ex}{0.4ex}} % ▂
\DeclareUnicodeCharacter{2583}{\rule{1ex}{0.6ex}} % ▃
\DeclareUnicodeCharacter{2584}{\rule{1ex}{0.8ex}} % ▄
\DeclareUnicodeCharacter{2585}{\rule{1ex}{1.0ex}} % ▅
\DeclareUnicodeCharacter{2586}{\rule{1ex}{1.2ex}} % ▆
\DeclareUnicodeCharacter{2587}{\rule{1ex}{1.4ex}} % ▇
\DeclareUnicodeCharacter{2588}{\rule{1ex}{1.6ex}} % █

